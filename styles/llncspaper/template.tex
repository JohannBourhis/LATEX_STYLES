\documentclass[runningheads,orivec,oribibl]{llncs}

\usepackage{lncspaper}

\spnewtheorem{assumption}{Assumption}{\bfseries}{\itshape}
\def\algorithmautorefname{Algorithm}
\def\assumptionautorefname{Assumption}

% A draft watermark is nice while the paper is in preparation
% but it interferes with syncTeX
\usepackage{draftwatermark}
\SetWatermarkScale{8}
\SetWatermarkLightness{.97}

% For the Cahier du GERAD.
%\renewcommand{\year}{2020}     % If you don't want the current year.
\newcommand{\cahiernumber}{00}  % Insert your Cahier du GERAD number.

% For debugging.
%\usepackage{showframe}

% For final version.
%\usepackage{butterma}
%\idline{J.~Doe and E.~Muster (Eds.): Perfect Publishing, LNCS 9999}
%\setcounter{page}{101}

% if you have landscape tables
%\usepackage[figuresright]{rotating}

\title{%
  Your Title Should Be Short and to the Point
}
\titlerunning{%
  Short and to the Point  % will appear in header
}
\author{
  Dominique Orban\inst{1,2}%
  \thanks{Research partially supported by NSERC Discovery Grant 299010-04.}
}
\institute{
  Department of Mathematics and Industrial Engineering,
  \'Ecole Polytechnique,
  Montr\'eal, QC, Canada.
  \and
  GERAD, Montr\'eal, QC, Canada.
  \mailto{dominique.orban@gerad.ca}
}
\authorrunning{Orban}

% Meta-information for the PDF file generated.
\pdfinfo{/Author (Dominique Orban)
         /Title (Insert Title Here)
         /Keywords (keyword1, keyword2, keyword3)}

% For final version, set paper format.
%\pdfpagesattr{/CropBox [92 112 523 778]} % LNCS page: 152x235 mm

\begin{document}

\linenumbers

\pagestyle{myheadings}

\maketitle
\thispagestyle{mytitlepage}   % Cahier du GERAD.
%\thispagestyle{electronic}   % Published version.

\begin{abstract}
  The abstract is reserved to an expert audience.
  It should clearly state what types of problems you consider, what your method is, and why it is novel.
  The abstract should also summarize your main theoretical results, say a few words about implementation, and sum up your findings from numerical experiments.
  The abstract should be written last.
  \smarttodo{write abstract}
\end{abstract}

\keywords{Keyword1, keyword2, keyword3}

% Pour le cahier du GERAD.
%\begin{resume}
%
%\end{resume}
%\textbf{Mots cl\'es :}

%========================================================================

\section{Introduction}
\label{sec:introduction}

We consider the problem
\begin{equation}
  \label{eq:unc}
  \minimize{x \in \R^n} \ f(x),
\end{equation}
where \(f : \R^n \to \R\) is continuously differentiable.

\subsection*{Related Research}

Cite the most relevant references, summarize what their approach and contributions are, and indicate how your method differs.

\subsection*{Notation}

We use Householder notation throughout: capital Latin letters such as \(A\), \(B\), and \(H\), represent matrices, lowercase Latin letters such as \(s\), \(x\), and \(y\) represent vectors in \(\R^n\), and lowercase greek letter such as \(\alpha\), \(\beta\) and \(\gamma\) represent scalars.

%========================================================================

%% this structure is only an example

\section{Background and Assumptions}
\label{sec:background}

\begin{assumption}
  \label{ass:f-C1}
  The function \(f\) is continuously differentiable over \(\R^n\).
\end{assumption}

%========================================================================

\section{Complete Algorithm}
\label{sec:algorithm}

We summarize the complete process as \autoref{alg:name-of-your-algorithm}.

\begin{algorithm}[t]
  \caption{Give a short name or description of the algorithm}
  \label{alg:name-of-your-algorithm}
  \begin{algorithmic}[1]
    \Require \(x_0 \in \R^n\)

    \State choose \(\epsilon > 0\), set \(k = 0\)
      \Comment{change as necessary}

    \For{$k = 1, 2, \ldots$}

      \State

      \State

    \EndFor
  \end{algorithmic}
\end{algorithm}

%========================================================================

\section{Convergence Analysis}
\label{sec:convergence}

\begin{lemma}
  Lemmas are used for preliminary results.
\end{lemma}

\begin{proposition}
  A proposition is an important result established by someone else.
\end{proposition}

\begin{theorem}
  Theorems are used to emphasize important results that you established, and distinguish them from the results in Propositions.

  Let \autoref{ass:f-C1} be satisfied.
  Then \(\dots\)
\end{theorem}

\begin{proof}
  Here is the proof.
  \qed
\end{proof}

%========================================================================

\section{Implementation and Numerical Results}
\label{sec:implementation}

\begin{itemize}
  \item describe implementation;
  \item brief explanation of the packages used;
  \item values of algorithmic parameters, choice of accuracy tolerances, stopping conditions, etc.;
  \item description and size of test problems;
  \item etc.
\end{itemize}

%========================================================================

\section{Discussion}
\label{sec:discussion}

A discussion is better than conclusions.
Contrast your finding with those from the literature and justify the statements you made in the abstract and introduction.
Finish with a brief statement of future work.

%========================================================================

\section{Notes}
\label{sec:notes}

You can keep notes in this section and remove the section altogether when you are ready to submit.

Using Natbib with author-year style, you can work citations into your sentences.
For example, \cite{wright-orban-2002} study the existence of the central path under the Mangasarian-Fromovitz constraint qualification.

%========================================================================

%% The Appendices part is started with the command \appendix;
%% appendix sections are then done as normal sections
%% \appendix

%% \section{}
%% \label{}

%% References
%% We use Natbib, which is accepted by SIAM and Springer journals
%% see the cheat sheet: http://merkel.texture.rocks/Latex/natbib.php

% \cite is the same as \citet
% \citet{jon90}	                  -->    	Jones et al. (1990)
% \citet[chap. 2]{jon90}	        -->    	Jones et al. (1990, chap. 2)
% \citep{jon90}	                  -->    	(Jones et al., 1990)
% \citep[chap. 2]{jon90}	        -->    	(Jones et al., 1990, chap. 2)
% \citep[see][]{jon90}	          -->    	(see Jones et al., 1990)
% \citep[see][chap. 2]{jon90}	    -->    	(see Jones et al., 1990, chap. 2)
% \citet*{jon90}	                -->    	Jones, Baker, and Williams (1990)
% \citep*{jon90}	                -->    	(Jones, Baker, and Williams, 1990)


%% Following citation commands can be used in the body text:
%% Usage of \cite is as follows:
%%   \cite{key}         ==>>  [#]
%%   \cite[chap. 2]{key} ==>> [#, chap. 2]
%%

%% References with BibTeX database:
% - each reference should have a DOI
% - use the strings provided for the journal name

\bibliographystyle{abbrvnat}
\bibliography{abbrv}
\bibliography{\jobname}

\newpage

\hypertarget{contents}{}  % so clicking on [toc] in the header leads here
\tableofcontents
\listoftodos

\end{document}
